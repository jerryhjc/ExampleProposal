%!TEX root = proposal.tex
%\addbibresource{bibliography.bib}
% ---------------------------------------------------------------------------
\section{Related Work}\label{sect:relatedWork}
% ---------------------------------------------------------------------------
To the best of my knowledge, no one else has proposed the use of a programming model as high-level as Galois for the 
purposes of HLS. I believe that this is also the first work implementing transactional memory semantics for HLS. 
However, previous work has explored the notion of using a vertex-centric graph language for HLS. 

\subsection{GraphGen for CoRAM}

Nurvitadhi et al. has proposed GraphGen for CoRAM \cite{graphGenCoRAM}, a vertex-centric graph language targeting the 
CoRAM \cite{coram} FPGA memory architecture. However, due to the vertex-centric nature of the input language, work 
scheduling and conflict detection can be done statically. GraphGen as an input language is more restrictive and 
frequently lower performance than the more complex Galois programming model. GraphGen for CoRAM proposes a few 
optimizations: double buffering, coalescing, pipeline parallelism, and multiple read ports. Galois for HLS implements 
pipeline parallelism in the Galois HLS compiler and the other optimizations and more in the microarchitecture.
