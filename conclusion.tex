%!TEX root = proposal.tex
% ---------------------------------------------------------------------------
\section{Conclusion}\label{sect:conclusion}
% ---------------------------------------------------------------------------
I propose the use of the parallel graph language Galois for FPGA, which presents a high-level transactional programming 
model, allowing the user to describe the fine-grained task parallelism available in amorphous data-parallel graph algorithms to be more easily 
extracted by the compiler and Galois runtime. By combining the high-level transactional semantics of Galois with 
the automatic Galois HLS compiler and Galois microarchitecture that supports dynamic scheduling and synchronization, my tool can simultaneously improve 
quality of results while decreasing programmer effort. To achieve these goals, I will augment an existing HLS compiler 
to support the Galois unordered \textbf{foreach} loop iterator, and design a set of Galois library components and 
Galois microarchitectural template. The compiler will emit custom pipelined engines and connect them to my proposed Galois 
microarchitecture. Given the scope of the project, I will first focus on Galois programs that do not modify the structure of 
the input graph, but will later experiment with a simple memory allocator for mutable graphs. 
I will also not be focusing on preserving atomicity across \textbf{foreach} loop iterations, as that 
is the thesis project of my labmate Xiaoyu Ma. Preliminary results for SSSP demonstrate that my proposed Galois 
microarchitecture is promising and should be effective for accelerating other Galois programs.
